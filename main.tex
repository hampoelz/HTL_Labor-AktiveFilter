\documentclass[a4paper]{hitec}
\settextfraction{0.95}      % reduce left margin

\usepackage{styles/main}
\usepackage{styles/custom}

\labtitle{Aktive Filter}

\author{Rene Hampölz, Gruppe 6}
\company{HTBLA Weiz, 5BHET}
\date{19. September 2022}

\begin{document}

\maketitletoc
\clearpage

\section{Einführung}

Es soll ein aktiver \textit{(Butterworth-)} Filter der zweiten Ordnung über der Sallen-Key-Schaltung dimensiniert und aufgebaut werden. Des weiteren soll ein passiver Filter der zweiten Ordnung Simuliert werden.

\begin{sagesilent}
    f_c = 1000
    R = 100
\end{sagesilent}

Angaben: $f_c = \SIvar{f_c}{\hertz}$, $R_1 = R_2 = \SIvar{R}{\ohm}$, $C_1 = C_2$

\section{Passiver Tiefpass-Filter}

\subsection{Schaltung \textit{(RC-Glied)}}

\begin{multicols}{2}
    \subsubsection{Erste Ordnung}

    \begin{circuitikz}
        \coordinate (in+) at (0,0);
        \coordinate (in-) at (0,-2);

        \draw
        (in+) to[R=$R$,o-] ++(2,0) coordinate (aux1)
        to[short,-o] ++(1,0) coordinate (out+)

        (aux1) to[C=$C$,*-*] (aux1 |- in-)

        (in-) to[short,o-o] (in- -| out+) coordinate (out-)

        (in+) to[open,name=inV,v] (in-)
        (out+) to[open,name=outV,v^] (out-)
        ;

        \voltage{inV}{$U_e$}
        \voltage{outV}{$U_a$}
    \end{circuitikz}

    \medskip
    
    Für die Grenzfrequenz gilt:

    \vspace{-1em}
    \begin{flalign*}
        f_c = \frac{1}{2\pi \cdot R \cdot C} &&
        \text{bzw.} &&
        w_c = \frac{f_c}{R \cdot C} &&
    \end{flalign*}
    
    \columnbreak
    \subsubsection{Zweite Ordnung}

    \begin{circuitikz}
        \coordinate (in+) at (0,0);
        \coordinate (in-) at (0,-2);

        \draw
        (in+) to[R=$R_1$,o-] ++(2,0) coordinate (aux1)
        to[C=$C_1$,*-*] (aux1 |- in-)

        (aux1) to[R=$R_2$] ++(2,0) coordinate (aux2)
        to[short,-o] ++(1,0) coordinate (out+)

        (aux2) to[C=$C_2$,*-*] (aux2 |- in-)
        
        (in-) to[short,o-o] (in- -| out+) coordinate (out-)

        (in+) to[open,name=inV,v] (in-)
        (out+) to[open,name=outV,v^] (out-)
        ;

        \voltage{inV}{$U_e$}
        \voltage{outV}{$U_a$}
    \end{circuitikz}

    \medskip

    Für die Grenzfrequenz gilt:

    \vspace{-1em}
    \begin{flalign*}
        f_c = \frac{1}{2\pi \sqrt{R_1 R_2 C_1 C_2}} &&
        \text{bzw.} &&
        w_c = \frac{f_c}{\sqrt{R_1 R_2 C_1 C_2}} &&
    \end{flalign*}
\end{multicols}

\subsection{Korrektur der Grenzfrequenz}

Ab Filter der 2. Ordnung entspricht die Grenzfrequenz $f_c$ nicht mehr der Verstärkung von $-3dB$. Deshalb muss die Grenzfrequenz für die gewünschte Filter-Ordnung $n$ korrigiert werden:

\vspace{-1em}
\begin{align*}
    f_{(-3dB)} = f_{c} \sqrt{2^{(\frac{1}{n})}-1} &&
    \text{bzw.} &&
    w_{(-3dB)} = w_{c} \sqrt{2^{(\frac{1}{n})}-1} &&
\end{align*}

\clearpage

\section{Aktiver Tiefpass-Filter}

\subsection{Schaltung \textit{(Sallen-Key-Topologie)}}

\begin{figure}[h]
    \centering
    \begin{circuitikz}
        \coordinate (in+) at (0,0);
        \coordinate (in-) at (0,-5);

        \draw
        (in+) to[R=$R_1$,o-*] ++(2,0) coordinate (aux1)
        to[R=$R_2$] ++(2,0) coordinate (aux2)
        to[short,*-] ++(0,-2) coordinate (aux3)
        to[C=$C_2$,-*] (aux3 |- in-)

        (aux2) -- ++(1,0) node[op amp,noinv input up,anchor=+] (OPV) {$OPV$}
        (OPV.-) -- ++(0,-1) coordinate (aux4) to[R=$R_4$,-*] (aux4 -| OPV.out)
        (OPV.out) -- ++(0,-2) coordinate (aux5) to[R=$R_3$,-*] (aux5 |- in-)

        (aux1) -- ++(0,2) coordinate (aux6) to[C=$C_1$] (aux6 -| OPV.out) -- (OPV.out)

        (OPV.out) to[short,-o] ++(1,0) coordinate (out+)
        (in-) to[short,o-o] (in- -| out+) coordinate (out-)

        (in+)   to[open,name=inV,v] (in-)
        (out+)  to[open,name=outV,v^] (out-)
        ;

        \voltage{inV}{$U_e$}
        \voltage{outV}{$U_a$}
    \end{circuitikz}
\end{figure}

\subsubsection{Verstärkungsfaktor $K$}

\begin{align*}
    A_0 = K &= 1 + \frac{R_4}{R_3} \\
    K &= 3 - \frac{1}{Q} \tag*{bei K >= 3 ist der Filter instabil!}
\end{align*}

\subsection{Filter-Polynom \textit{(Butterworth)}}

Beim Butterworth-Polynom wird ein Signal an der Grenzfrequenz auf das $\frac{1}{\sqrt{2}} = 0,707$-fache des ursprünglichen Signals abgeschwächt. Daraus ergibt sich die Güte $Q$, die Gleichspannungsverstärkung $K$, sowie die Übertragungsfunktion $\frac{U_a}{U_e}$:

\begin{sagesilent}
    Q = 1/sqrt(2)
    K = 3 - sqrt(2)
\end{sagesilent}

\begin{subequations}
    \begin{equation*}
        Q = \frac{1}{\sqrt{2}} = \var{Q, 3}
    \end{equation*}
    \begin{equation*}
        K = 3 - \sqrt{2} = \var{K, 3}
    \end{equation*}
\end{subequations}

\begin{equation*}
    \frac{U_a}{U_e} = \frac{K \cdot \frac{1}{R_1 R_2 C_1 C_2}}{s^2+s[\frac{R_1 C_2 + R_2 C_2 + R_1 C_1 \cdot (1-k)}{R_1 R_2 C_1 C_2}] + \frac{1}{R_1 R_2 C_1 C_2}}
    \tag*{mit \quad $s = j \cdot w$}
\end{equation*}

Da in dieser Übung $R_1 = R_2$ und $C_1 = C_2$ entspricht, kann das Polynom vereinfacht werden. Durch umformen und einsetzten von $w_c = \frac{f_c}{\sqrt{R_1 R_2 C_1 C_2}}$, wird das Polynom weiter vereinfacht:

\begin{equation*}
    \frac{U_a}{U_e} = \frac{K \cdot w_c^2}{s^2 + s \cdot w_c \cdot (3-K) + w_c^2}
\end{equation*}

\clearpage

\section{Passiver vs. Aktiver Filter}

Im Gegensatz zum passiven Filter haben aktive Filter höhere Eingangs- und niedrigere Ausgangsimpedanzen. Aufgrund des OPVs benutzen aktive Filter \underline{nicht} die Energie des Signals sondern benötigen eine externe Stromversorgung.

\section{Berechnungen}

\subsection{Korrektur der Grenzfrequenz}

Da es sich bei der Sallen-Key-Topologie um Filter der zweiten Ordnung handelt, muss die Grenzfrequenz $f_c$ korrigiert werden, um weiterhin eine Verstärkung von $-3dB$ zu erreichen:

\begin{sagesilent}
    f_3db = f_c
    f_c = var('f_c')

    equ1 = f_3db == f_c * sqrt(2^(1/2) - 1)
    f_c = solve(equ1, f_c, solution_dict=True)[0][f_c]
    f_c = round(f_c, ndigits=2)
\end{sagesilent}

\begin{align*}
    f_{(-3dB)} &= f_c \sqrt{2^{\frac{1}{n}} - 1} \tag*{($f_{(-3dB)} = \SIvar{f_3db}{\hertz}$)} \\
    \var{f_3db} &= f_c \cdot \sqrt{2^{\frac{1}{2}} - 1} \\
    \rightarrow f_c &= \SIvar{f_c}{\hertz}
\end{align*}

\subsection{Kapazität $C$}

Um nun den Bauteilwert für die Kapazität $C$ zu berechnen, wird die Formel der Grenzfrequenz $f_c$ für die zweite Ordnung umgeformt:

\begin{sagesilent}
    C = 1/(2*pi*R*f_c)
\end{sagesilent}

\begin{align*}
    f_c &= \frac{1}{2\pi \sqrt{R_1 R_2 C_1 C_2}} \tag*{($R_1 = R_2$, $C_1 = C_2$)} \\
    f_c &= \frac{1}{2\pi \cdot R \cdot C} \\
    C &= \frac{1}{2\pi \cdot R \cdot f_c} \tag*{($R = \SIvar{R}{\ohm}$)} \\
    C &= \frac{1}{2\pi \cdot \var{R} \cdot \var{f_c}} \\
    C &= \SIvar{C, 2}{\farad}
\end{align*}

\subsection{Widerstände $R_3$ und $R_4$}

Die letzten zwei Bauteilwerte, für die Widerstände $R_3$ und $R_4$, werden über das Widerstandsverhältnis des Verstärkungsfaktors $K$ ausgerechnet:

\begin{sagesilent}
    R_3 = 10000
\end{sagesilent}

\begin{align*}
    K &= 1 + \frac{R_4}{R_3} \\
    \frac{R_4}{R_3} &= K - 1 \tag*{($K = \var{K, 3}$)} \\
    \frac{R_4}{R_3} &= \var{K-1, 3} = \frac{\var{K-1, 3}}{1} \tag*{(Angenommen $R_3 = \SIvar{R_3}{\ohm}$)} \\
    \frac{R_4}{R_3} &\sim \frac{\SIvar{(K-1)*R_3, 0}{\ohm}}{\SIvar{R_3}{\ohm}}
\end{align*}

\section{Simulationen}

\subsection{Passiver Tiefpass-Filter \tsizesmall{\textit{(2. Ordnung)}}}

\begin{figure}[H]
    \centering
    \includesvg[width=\textwidth]{images/01 Passiver Tiefpass-Filter.svg}
    \caption{Amplituden- und Phasengang des Simulierten passiven Tiefpass-Filters}
\end{figure}

\subsection{Aktiver Tiefpass-Filter \tsizesmall{\textit{(2. Ordnung)}}}

\begin{figure}[H]
    \centering
    \includesvg[width=\textwidth]{images/02 Aktiver Tiefpass-Filter.svg}
    \caption{Amplituden- und Phasengang des Simulierten aktiven Tiefpass-Filters}
\end{figure}

\clearpage

\section{Auswertung}

\subsection{Messwerte}

Aufgrund eines falsch gewählten OPVs konnten aus der aufgebauten Schaltung keine Werte zur Auswertung entnommen werden. Stattdessen wurde der aktive Filter ebenfalls Simuliert.

\subsection{Verwendete Geräte}

\medskip

\begin{devicelist}
    \devicecat{Frequenzgenerator}{1}{
        \device{ET-MTL1-FG01}{U_e}
    }
    \devicecat{Widerstands-Dekade}{2}{
        \device{ET-MTL1-RD09}{R_1}
        \device{ET-MTL1-RD14}{R_2}
        \device{ET-MTL1-RD28}{R_3}
        \device{ET-MTL1-RG26}{R_4}
    }
    \devicecat{Kapazitäts-Dekade}{3}{
        \device{ET-MTL1-CD11}{C_1}
        \device{ET-MTL1-CD18}{C_2}
    }
    \devicecat{Oszilloskop}{1}{
        \device{ET-MTL1-OS04}{U_e, U_a}
    }
    \devicecat{Operationsverstärker}{1}{
        \device{OP27}{OPV}
    }
\end{devicelist}

\vfill

\IncludeHistoryTimeline

\end{document}